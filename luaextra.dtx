% \iffalse meta-comment
%
% Copyright (C) 2009 by PRAGMA ADE / ConTeXt Development Team
%
% See ConTeXt's mreadme.pdf for the license.
%
% This work consists of the main source file luaextra.dtx
% and the derived file luaextra.lua.
%
% Unpacking:
%    tex luatextra.dtx
%
% Documentation:
%    pdflatex luaextra.dtx
%
%    The class ltxdoc loads the configuration file ltxdoc.cfg
%    if available. Here you can specify further options, e.g.
%    use A4 as paper format:
%       \PassOptionsToClass{a4paper}{article}
%
%
%
%<*ignore>
\begingroup
  \def\x{LaTeX2e}%
\expandafter\endgroup
\ifcase 0\ifx\install y1\fi\expandafter
         \ifx\csname processbatchFile\endcsname\relax\else1\fi
         \ifx\fmtname\x\else 1\fi\relax
\else\csname fi\endcsname
%</ignore>
%<*install>
\input docstrip.tex
\Msg{************************************************************************}
\Msg{* Installation}
\Msg{* Package: luaextra 2010/01/11 v0.92 Lua additional functions.}
\Msg{************************************************************************}

\keepsilent
\askforoverwritefalse

\let\MetaPrefix\relax

\preamble
This is a generated file.

Copyright (C) 2009 by PRAGMA ADE / ConTeXt Development Team

See ConTeXt's mreadme.pdf for the license.

This work consists of the main source file luaextra.dtx
and the derived file luaextra.lua.

\endpreamble

% The following hacks are to generate a lua file with lua comments starting by
% -- instead of %%

\def\MetaPrefix{-- }

\def\luapostamble{%
  \MetaPrefix^^J%
  \MetaPrefix\space End of File `\outFileName'.%
}

\def\currentpostamble{\luapostamble}%

\generate{%
  \usedir{tex/luatex/luatextra}%
  \file{luaextra.lua}{\from{luaextra.dtx}{lua}}%
}

\obeyspaces
\Msg{************************************************************************}
\Msg{*}
\Msg{* To finish the installation you have to move the following}
\Msg{* files into a directory searched by TeX:}
\Msg{*}
\Msg{*     luaextra.lua}
\Msg{*}
\Msg{* Happy TeXing!}
\Msg{*}
\Msg{************************************************************************}

\endbatchfile
%</install>
%<*ignore>
\fi
%</ignore>
%<*driver>
\NeedsTeXFormat{LaTeX2e}
\ProvidesFile{luaextra.drv}
  [2010/01/11 v0.92 Lua additional functions.]
\documentclass{ltxdoc}
\EnableCrossrefs
\CodelineIndex
\begin{document}
  \DocInput{luaextra.dtx}%
\end{document}
%</driver>
% \fi
% \CheckSum{0}
%
% \CharacterTable
%  {Upper-case    \A\B\C\D\E\F\G\H\I\J\K\L\M\N\O\P\Q\R\S\T\U\V\W\X\Y\Z
%   Lower-case    \a\b\c\d\e\f\g\h\i\j\k\l\m\n\o\p\q\r\s\t\u\v\w\x\y\z
%   Digits        \0\1\2\3\4\5\6\7\8\9
%   Exclamation   \!     Double quote  \"     Hash (number) \#
%   Dollar        \$     Percent       \%     Ampersand     \&
%   Acute accent  \'     Left paren    \(     Right paren   \)
%   Asterisk      \*     Plus          \+     Comma         \,
%   Minus         \-     Point         \.     Solidus       \/
%   Colon         \:     Semicolon     \;     Less than     \<
%   Equals        \=     Greater than  \>     Question mark \?
%   Commercial at \@     Left bracket  \[     Backslash     \\
%   Right bracket \]     Circumflex    \^     Underscore    \_
%   Grave accent  \`     Left brace    \{     Vertical bar  \|
%   Right brace   \}     Tilde         \~}
%
% \GetFileInfo{luaextra.drv}
%
% \title{The \textsf{luaextra} package}
% \date{2010/01/11 v0.92}
% \author{Elie Roux \\ \texttt{elie.roux@telecom-bretagne.eu}}
%
% \maketitle
%
% \begin{abstract}
% Additional lua functions taken from the libs of Con\TeX t. For an
% introduction on this package (among others), please refer to the document
% \texttt{luatex-reference.pdf}.
% \end{abstract}
%
% \section{Overview}
%
% Lua is a very minimal language, and it does not have a lot of built-in
% functions. Some functions will certainly be needed by a lot of packages.
% Instead of making each of them implement these functions, the aim of this
% file is to provide a minimal set of functions. All functions are taken from
% Con\TeX t libraries.
%
% There are some differences with the Con\TeX t funtions though, especially on
% names: for example the \texttt{file.*} funtions are renamed in
% \texttt{fpath.*}. It seems more logical as they deal with file paths, not
% files. Also the \texttt{file.is\_readable} and \texttt{file.is\_writable}
% are renamed \texttt{lfs.is\_readable} and \texttt{lfs.is\_writable}.
%
% If you use a function you think is missing in this file, please tell the
% maintainer.
%
% \texttt{Warning:} Even if the names will certainly remain the same, some
% implementations may differ, and some functions might appear or dissapear. As
% Lua\TeX\ is not stable, this file is not neither.
%
% All functions are described in this document, but the one of the functions
% you'll use most will certainly be \texttt{table.serialize} (also named
% \texttt{table.tostring}) that takes a table and returns an intented string
% describing the table. It describes the table so that Lua\TeX\ can read it
% again as a table. You can do a lot of things with this functions, like
% printing a table for debugging, or saving a table into a file. Functions are
% also converted into bytecode to be saved.
%
% \section{\texttt{luaextra.lua}}
%
% \iffalse
%<*lua>
% \fi
%
%    \begin{macrocode}
do
    local luaextra_module = {
        name          = "luaextra",
        version       = 0.92,
        date          = "2010/01/11",
        description   = "Lua additional functions.",
        author        = "Hans Hagen, PRAGMA-ADE, Hasselt NL & Elie Roux",
        copyright     = "PRAGMA ADE / ConTeXt Development Team",
        license       = "See ConTeXt's mreadme.pdf for the license",
    }
    if luatextra then
        luatextra.provides_module(luaextra_module)
    end
end
%    \end{macrocode}
%
%    \begin{macro}{string:stripspaces}
%
%    A function to strip the spaces at the beginning and at the end of a
%    string.
%
%    \begin{macrocode}

function string:stripspaces()
    return (self:gsub("^%s*(.-)%s*$", "%1"))
end

%    \end{macrocode}
%
%    \end{macro}
%    \begin{macro}{string.is boolean}
%
%    If the argument is a string describing a boolean, this function returns
%    the boolean, otherwise it retuns nil.
%
%    \begin{macrocode}

function string.is_boolean(str)
    if type(str) == "string" then
        if str == "true" or str == "yes" or str == "on" or str == "t" then
            return true
        elseif str == "false" or str == "no" or str == "off" or str == "f" then
            return false
        end
    end
    return nil
end

%    \end{macrocode}
%
%    \end{macro}
%    \begin{macro}{string.is number}
%
%    Returns true if the argument string is a number.
%
%    \begin{macrocode}

function string.is_number(str)
    return str:find("^[%-%+]?[%d]-%.?[%d+]$") == 1
end

%    \end{macrocode}
%
%    \end{macro}
%    \begin{macro}{lpeg.space and lpeg.newline}
%
%    Two small helpers for \texttt{lpeg}, that will certainly be widely used:
%    spaces and newlines.
%
%    \begin{macrocode}

lpeg.space    = lpeg.S(" \t\f\v")
lpeg.newline  = lpeg.P("\r\n") + lpeg.P("\r") +lpeg.P("\n")

%    \end{macrocode}
%
%    \end{macro}
%    \begin{macro}{table.fastcopy}
%
%    A function copying a table fastly.
%
%    \begin{macrocode}

if not table.fastcopy then do

    local type, pairs, getmetatable, setmetatable =
        type, pairs, getmetatable, setmetatable

    local function fastcopy(old) -- fast one
        if old then
            local new = { }
            for k,v in pairs(old) do
                if type(v) == "table" then
                    new[k] = fastcopy(v) -- was just table.copy
                else
                    new[k] = v
                end
            end
            local mt = getmetatable(old)
            if mt then
                setmetatable(new,mt)
            end
            return new
        else
            return { }
        end
    end

    table.fastcopy = fastcopy

end end

%    \end{macrocode}
%
%    \end{macro}
%    \begin{macro}{table.copy}
%
%    A function copying a table in more cases than fastcopy, for example when
%    a key is a table.
%
%    \begin{macrocode}

if not table.copy then do

    local type, pairs, getmetatable, setmetatable = type, pairs, getmetatable, setmetatable

    local function copy(t, tables) -- taken from lua wiki, slightly adapted
        tables = tables or { }
        local tcopy = {}
        if not tables[t] then
            tables[t] = tcopy
        end
        for i,v in pairs(t) do -- brrr, what happens with sparse indexed
            if type(i) == "table" then
                if tables[i] then
                    i = tables[i]
                else
                    i = copy(i, tables)
                end
            end
            if type(v) ~= "table" then
                tcopy[i] = v
            elseif tables[v] then
                tcopy[i] = tables[v]
            else
                tcopy[i] = copy(v, tables)
            end
        end
        local mt = getmetatable(t)
        if mt then
            setmetatable(tcopy,mt)
        end
        return tcopy
    end

    table.copy = copy

end end

%    \end{macrocode}
%
%    \end{macro}
%    \begin{macro}{table.serialize}
%
%    A bunch of functions leading to \texttt{table.serialize}.
%
%    \begin{macrocode}

function table.sortedkeys(tab)
    local srt, kind = { }, 0 -- 0=unknown 1=string, 2=number 3=mixed
    for key,_ in pairs(tab) do
        srt[#srt+1] = key
        if kind == 3 then
            -- no further check
        else
            local tkey = type(key)
            if tkey == "string" then
            --  if kind == 2 then kind = 3 else kind = 1 end
                kind = (kind == 2 and 3) or 1
            elseif tkey == "number" then
            --  if kind == 1 then kind = 3 else kind = 2 end
                kind = (kind == 1 and 3) or 2
            else
                kind = 3
            end
        end
    end
    if kind == 0 or kind == 3 then
        table.sort(srt,function(a,b) return (tostring(a) < tostring(b)) end)
    else
        table.sort(srt)
    end
    return srt
end

do
    table.serialize_functions = true
    table.serialize_compact   = true
    table.serialize_inline    = true

    local function key(k)
        if type(k) == "number" then -- or k:find("^%d+$") then
            return "["..k.."]"
        elseif noquotes and k:find("^%a[%a%d%_]*$") then
            return k
        else
            return '["'..k..'"]'
        end
    end

    local function simple_table(t)
        if #t > 0 then
            local n = 0
            for _,v in pairs(t) do
                n = n + 1
            end
            if n == #t then
                local tt = { }
                for i=1,#t do
                    local v = t[i]
                    local tv = type(v)
                    if tv == "number" or tv == "boolean" then
                        tt[#tt+1] = tostring(v)
                    elseif tv == "string" then
                        tt[#tt+1] = ("%q"):format(v)
                    else
                        tt = nil
                        break
                    end
                end
                return tt
            end
        end
        return nil
    end

    local function serialize(root,name,handle,depth,level,reduce,noquotes,indexed)
        handle = handle or print
        reduce = reduce or false
        if depth then
            depth = depth .. " "
            if indexed then
                handle(("%s{"):format(depth))
            else
                handle(("%s%s={"):format(depth,key(name)))
            end
        else
            depth = ""
            local tname = type(name)
            if tname == "string" then
                if name == "return" then
                    handle("return {")
                else
                    handle(name .. "={")
                end
            elseif tname == "number" then
                handle("[" .. name .. "]={")
            elseif tname == "boolean" then
                if name then
                    handle("return {")
                else
                    handle("{")
                end
            else
                handle("t={")
            end
        end
        if root and next(root) then
            local compact = table.serialize_compact
            local inline  = compact and table.serialize_inline
            local first, last = nil, 0 -- #root cannot be trusted here
            if compact then
              for k,v in ipairs(root) do -- NOT: for k=1,#root do (why)
                    if not first then first = k end
                    last = last + 1
                end
            end
            for _,k in pairs(table.sortedkeys(root)) do
                local v = root[k]
                local t = type(v)
                if compact and first and type(k) == "number" and k >= first and k <= last then
                    if t == "number" then
                        handle(("%s %s,"):format(depth,v))
                    elseif t == "string" then
                        if reduce and (v:find("^[%-%+]?[%d]-%.?[%d+]$") == 1) then
                            handle(("%s %s,"):format(depth,v))
                        else
                            handle(("%s %q,"):format(depth,v))
                        end
                    elseif t == "table" then
                        if not next(v) then
                            handle(("%s {},"):format(depth))
                        elseif inline then
                            local st = simple_table(v)
                            if st then
                                handle(("%s { %s },"):format(depth,table.concat(st,", ")))
                            else
                                serialize(v,k,handle,depth,level+1,reduce,noquotes,true)
                            end
                        else
                            serialize(v,k,handle,depth,level+1,reduce,noquotes,true)
                        end
                    elseif t == "boolean" then
                        handle(("%s %s,"):format(depth,tostring(v)))
                    elseif t == "function" then
                        if table.serialize_functions then
                            handle(('%s loadstring(%q),'):format(depth,string.dump(v)))
                        else
                            handle(('%s "function",'):format(depth))
                        end
                    else
                        handle(("%s %q,"):format(depth,tostring(v)))
                    end
                elseif k == "__p__" then -- parent
                    if false then
                        handle(("%s __p__=nil,"):format(depth))
                    end
                elseif t == "number" then
                    handle(("%s %s=%s,"):format(depth,key(k),v))
                elseif t == "string" then
                    if reduce and (v:find("^[%-%+]?[%d]-%.?[%d+]$") == 1) then
                        handle(("%s %s=%s,"):format(depth,key(k),v))
                    else
                        handle(("%s %s=%q,"):format(depth,key(k),v))
                    end
                elseif t == "table" then
                    if not next(v) then
                        handle(("%s %s={},"):format(depth,key(k)))
                    elseif inline then
                        local st = simple_table(v)
                        if st then
                            handle(("%s %s={ %s },"):format(depth,key(k),table.concat(st,", ")))
                        else
                            serialize(v,k,handle,depth,level+1,reduce,noquotes)
                        end
                    else
                        serialize(v,k,handle,depth,level+1,reduce,noquotes)
                    end
                elseif t == "boolean" then
                    handle(("%s %s=%s,"):format(depth,key(k),tostring(v)))
                elseif t == "function" then
                    if table.serialize_functions then
                        handle(('%s %s=loadstring(%q),'):format(depth,key(k),string.dump(v)))
                    else
                        handle(('%s %s="function",'):format(depth,key(k)))
                    end
                else
                    handle(("%s %s=%q,"):format(depth,key(k),tostring(v)))
                --  handle(('%s %s=loadstring(%q),'):format(depth,key(k),string.dump(function() return v end)))
                end
            end
            if level > 0 then
                handle(("%s},"):format(depth))
            else
                handle(("%s}"):format(depth))
            end
        else
            handle(("%s}"):format(depth))
        end
    end

    function table.serialize(root,name,reduce,noquotes)
        local t = { }
        local function flush(s)
            t[#t+1] = s
        end
        serialize(root, name, flush, nil, 0, reduce, noquotes)
        return table.concat(t,"\n")
    end

    function table.tostring(t, name)
        return table.serialize(t, name)
    end

    function table.tohandle(handle,root,name,reduce,noquotes)
        serialize(root, name, handle, nil, 0, reduce, noquotes)
    end

    -- sometimes tables are real use (zapfino extra pro is some 85M) in which
    -- case a stepwise serialization is nice; actually, we could consider:
    --
    -- for line in table.serializer(root,name,reduce,noquotes) do
    --    ...(line)
    -- end
    --
    -- so this is on the todo list

    table.tofile_maxtab = 2*1024

    function table.tofile(filename,root,name,reduce,noquotes)
        local f = io.open(filename,'w')
        if f then
            local concat = table.concat
            local maxtab = table.tofile_maxtab
            if maxtab > 1 then
                local t = { }
                local function flush(s)
                    t[#t+1] = s
                    if #t > maxtab then
                        f:write(concat(t,"\n"),"\n") -- hm, write(sometable) should be nice
                        t = { }
                    end
                end
                serialize(root, name, flush, nil, 0, reduce, noquotes)
                f:write(concat(t,"\n"),"\n")
            else
                local function flush(s)
                    f:write(s,"\n")
                end
                serialize(root, name, flush, nil, 0, reduce, noquotes)
            end
            f:close()
        end
    end

end

%    \end{macrocode}
%
%    \end{macro}
%    \begin{macro}{table.tohash}
%
%    Returning a table with all values of the argument table as keys, and
%    \texttt{false} as values. This is what we will call a hash.
%
%    \begin{macrocode}

function table.tohash(t)
    local h = { }
    for _, v in pairs(t) do -- no ipairs here
        h[v] = true
    end
    return h
end

%    \end{macrocode}
%
%    \end{macro}
%    \begin{macro}{table.fromhash}
%
%    Returning a table built from a hash, with simple integer keys.
%
%    \begin{macrocode}

function table.fromhash(t)
    local h = { }
    for k, v in pairs(t) do -- no ipairs here
        if v then h[#h+1] = k end
    end
    return h
end

%    \end{macrocode}
%
%    \end{macro}
%    \begin{macro}{table.contains value}
%
%    A function returning true if the value \texttt{val} is in the table
%    \texttt{t}.
%
%    \begin{macrocode}

function table.contains_value(t, val)
    if t then
        for k, v in pairs(t) do
            if v==val then
                return true
            end
        end
    end
    return false
end

%    \end{macrocode}
%
%    \end{macro}
%    \begin{macro}{table.contains key}
%
%    A function returning true if the key \texttt{key} is in the table
%    \texttt{t}
%
%    \begin{macrocode}

function table.contains_key(t, key)
    if t then
        for k, v in pairs(t) do
            if k==key then
                return true
            end
        end
    end
    return false
end

%    \end{macrocode}
%
%    \end{macro}
%    \begin{macro}{table.value position}
%
%    A function returning the position of a value in a table. This will be
%    important to be able to remove a value.
%
%    \begin{macrocode}

function table.value_position(t, val)
    if t then
        local i=1
        for k, v in pairs(t) do
            if v==val then
                return i
            end
            i=i+1
        end
    end
    return 0
end

%    \end{macrocode}
%
%    \end{macro}
%    \begin{macro}{table.key position}
%
%    A function returning the position of a key in a table.
%
%    \begin{macrocode}

function table.key_position(t, key)
    if t then
        local i=1
        for k,v in pairs(t) do
            if k==key then
                return i
            end
            i = i+1
        end
    end
    return -1
end

%    \end{macrocode}
%
%    \end{macro}
%    \begin{macro}{table.remove value}
%
%    Removes the first occurence of a value from a table.
%
%    \begin{macrocode}

function table.remove_value(t, v)
    local p = table.value_position(t,v)
    if p ~= -1 then
        table.remove(t, table.value_position(t,v))
    end
end

%    \end{macrocode}
%
%    \end{macro}
%    \begin{macro}{table.remove key}
%
%    Removing a key from a table.
%
%    \begin{macrocode}

function table.remove_key(t, k)
    local p = table.key_position(t,k)
    if p ~= -1 then
        table.remove(t, table.key_position(t,k))
    end
end

%    \end{macrocode}
%
%    \end{macro}
%    \begin{macro}{table.is empty}
%
%    Returns true if a table is empty.
%
%    \begin{macrocode}

function table.is_empty(t)
    return not t or not next(t)
end

%    \end{macrocode}
%
%     \texttt{fpath} will contain all the file path manipulation functions.
%     Some functions certainly need a little update or cleanup...
%
%    \begin{macrocode}

fpath = { }

%    \end{macrocode}
%
%    \end{macro}
%    \begin{macro}{fpath.removesuffix}
%
%    A function to remove the suffix (extention) of a filename.
%
%    \begin{macrocode}

function fpath.removesuffix(filename)
    return filename:gsub("%.[%a%d]+$", "")
end

%    \end{macrocode}
%
%    \end{macro}
%    \begin{macro}{fpath.addsuffix}
%
%    A function adding a suffix to a filename, except if it already has one.
%
%    \begin{macrocode}

function fpath.addsuffix(filename, suffix)
    if not filename:find("%.[%a%d]+$") then
        return filename .. "." .. suffix
    else
        return filename
    end
end

%    \end{macrocode}
%
%    \end{macro}
%    \begin{macro}{fpath.replacesuffix}
%
%    A function replacing a suffix by a new one.
%
%    \begin{macrocode}

function fpath.replacesuffix(filename, suffix)
    if not filename:find("%.[%a%d]+$") then
        return filename .. "." .. suffix
    else
        return (filename:gsub("%.[%a%d]+$","."..suffix))
    end
end

%    \end{macrocode}
%
%    \end{macro}
%    \begin{macro}{fpath.dirname}
%
%    A function returning the directory of a file path.
%
%    \begin{macrocode}

function fpath.dirname(name)
    return name:match("^(.+)[/\\].-$") or ""
end

%    \end{macrocode}
%
%    \end{macro}
%    \begin{macro}{fpath.basename}
%
% A function returning the basename (the name of the file, without the directories) of a file path.
%
%    \begin{macrocode}

function fpath.basename(fname)
    if not fname then
        return nil
    end
    return fname:match("^.+[/\\](.-)$") or fname
end

%    \end{macrocode}
%
%    \end{macro}
%    \begin{macro}{fpath.nameonly}
%
%    Returning the basename of a file without the suffix.
%
%    \begin{macrocode}

function fpath.nameonly(name)
    return ((name:match("^.+[/\\](.-)$") or name):gsub("%..*$",""))
end

%    \end{macrocode}
%
%    \end{macro}
%    \begin{macro}{fpath.suffix}
%
%    Returns the suffix of a file name.
%
%    \begin{macrocode}

function fpath.suffix(name)
    return name:match("^.+%.([^/\\]-)$") or  ""
end

%    \end{macrocode}
%
%    \end{macro}
%    \begin{macro}{fpath.join}
%
%    A function joining any number of arguments into a complete path.
%
%    \begin{macrocode}

function fpath.join(...)
    local pth = table.concat({...},"/")
    pth = pth:gsub("\\","/")
    local a, b = pth:match("^(.*://)(.*)$")
    if a and b then
        return a .. b:gsub("//+","/")
    end
    a, b = pth:match("^(//)(.*)$")
    if a and b then
        return a .. b:gsub("//+","/")
    end
    return (pth:gsub("//+","/"))
end

%    \end{macrocode}
%
%    \end{macro}
%    \begin{macro}{fpath.split}
%
%    A function returning a table with all directories from a filename.
%
%    \begin{macrocode}

function fpath.split(str)
    local t = { }
    str = str:gsub("\\", "/")
    str = str:gsub("(%a):([;/])", "%1\001%2")
    for name in str:gmatch("([^;:]+)") do
        if name ~= "" then
            name = name:gsub("\001",":")
            t[#t+1] = name
        end
    end
    return t
end

%    \end{macrocode}
%
%    \end{macro}
%    \begin{macro}{fpath.normalize sep}
%
%    A function to change directory separators to canonical ones (\texttt{/}).
%
%    \begin{macrocode}

function fpath.normalize_sep(str)
    return str:gsub("\\", "/")
end

%    \end{macrocode}
%
%    \end{macro}
%    \begin{macro}{fpath.localize sep}
%
%    A function changing directory separators into local ones (\texttt{/} on
%    Unix, |\| on Windows).
%
%    \begin{macrocode}

function fpath.localize_sep(str)
    if os.type == 'windows' or os.type == 'msdos' then
        return str:gsub("/", "\\")
    else
        return str:gsub("\\", "/")
    end
end

%    \end{macrocode}
%
%    \end{macro}
%    \begin{macro}{lfs.is writable}
%
%    Returns true if a file is writable. This function and the following ones
%    are a bit too expensive, they should be made with |lfs.attributes|.
%
%    \begin{macrocode}

function lfs.is_writable(name)
    local f = io.open(name, 'w')
    if f then
        f:close()
        return true
    else
        return false
    end
end

%    \end{macrocode}
%
%    \end{macro}
%    \begin{macro}{lfs.is readable}
%
%    Returns true if a file is readable.
%
%    \begin{macrocode}

function lfs.is_readable(name)
    local f = io.open(name,'r')
    if f then
        f:close()
        return true
    else
        return false
    end
end

%    \end{macrocode}
%
%    \end{macro}
%    \begin{macro}{math.round}
%
%    Returns the closest integer.
%
%    \begin{macrocode}

if not math.round then
    function math.round(x)
        return math.floor(x + 0.5)
    end
end

%    \end{macrocode}
%
%    \end{macro}
%    \begin{macro}{math.div}
%
%    Returns the quotient of the euclidian division of n by m.
%
%    \begin{macrocode}

if not math.div then
    function math.div(n,m)
        return floor(n/m)
    end
end

%    \end{macrocode}
%
%    \end{macro}
%    \begin{macro}{math.mod}
%
%    Returns the remainder of the euclidian division of n by m.
%
%    \begin{macrocode}

if not math.mod then
    function math.mod(n,m)
        return n % m
    end
end

%    \end{macrocode}
%
% \end{macro}
%
% \iffalse
%</lua>
% \fi
% \Finale
\endinput
